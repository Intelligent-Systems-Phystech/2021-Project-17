\documentclass{article}
\usepackage[utf8]{inputenc}
\usepackage{hyperref}
\usepackage[T2A]{fontenc}	
\usepackage{amssymb}
\usepackage[english, russian]{babel}
\title{BCI: Выбор согласованных моделей для построения нейроинтерфейса}
\author{Кулаков Ярослав}
\date{February 2021}

\begin{document}

\maketitle

\section{Abstract}
Разрабатывается Brain-Computer Interface. В настоящее время все больше задач перекладывается на плечи машин, машинного обучения. В то же время наш собственный мозг остается неизученным и трудно прогнозируемым объектом. В данной работе рассматривается задача моделирования сигнала, снимаемого с конечностей при некой активности мозга, а также моделирования исходного сигнала головного мозга по движению частей тела. Сначала предлагается применить методы снижения размерности исходного пространства, так как точки сильно скоррелированы.

Новизна: Анализ прогноза и латентного пространства, получаемых парой гетерогенных моделей.

Эксперементальные результаты подтверждают, что предлагаемый метод улучшает качество предсказаний модели.

\section{Introduction}
В работе предлагается исследовать и сравнить разные модели машинного обучения - начиная от линейных и заканчивая сложными нейросетями. Оценивается качество, скорость, устойчивость и сложность пар разных моделей. Но прежде чем их применить решается задача снижения размерности исходного пространства. В статье (Катруца, Стрижов, 2017) даны обширные сравнения QPFS с LARS, Lasso, Stepwise, Ridge и отбор признаков с генетическим алгоритмом. Quadratic Programming Feature Selection (QPFS) превосходит конкурентов и этот метод можно адаптировать для нашей задачи. Так же проводится сравнение с методом PLS.  При решении задачи отображения в латентное пространство приходится одновременно оптимизировать две задачи - минимизировать корреляцию между признаками и максимизировать информативность 
признаков по отношению к таргету. Задача осложняется тем, что признаки и таргеты имеют разную природу. 
В результате получен устойчивый пайплан модулирования сигнала в конечности от карты активности мозга, состоящий из этапов:
\begin{itemize}
    \item Построение латентного пространства меньшей размерности, с минимальной корреляцией признаков и сохранившего максимум информации о таргет-сигнале.
    \item Применение уже в полученном пространстве лучшей модели.
     \item Восстановление обратной зависимости для предсказания активности мозга.
\end{itemize}

\section{Literature}
\renewcommand\labelitemi{$\textasteriskcentered$}
\begin{itemize}
\item  Яушев Ф.Ю., Исаченко Р.В., Стрижов В.В. Модели согласования скрытого пространства в задаче прогнозирования  Системы и средства информатики, 2021, 31(1). \href{http://strijov.com/papers/Isachenko2020CanonicCorrelation.pdf}{PDF}.
\item  Исаченко Р.В. Выбор модели декодирования сигналов в пространствах высокой размерности. Рукопись, 2021.\href{https://github.com/r-isachenko/PhDThesis/raw/master/doc/Isachenko2021PhDThesis.pdf}{PDF} 
\item  Исаченко Р.В. Выбор модели декодирования сигналов в пространствах высокой размерности. \href{https://github.com/r-isachenko/PhDThesis/raw/master/pres/Isachenko2020PhDThesisPres.pdf}{Слайды}.
\item  Isachenko R.V., Vladimirova M.R., Strijov V.V. Dimensionality reduction for time series decoding and forecasting problems // DEStech Transactions on Computer Science and Engineering, 2018, 27349 : 286-296. \href{http://strijov.com/papers/IsachenkoVladimirova2018PLS.pdf}{PDF}. 
\item  Isachenko R.V., Strijov V.V. Quadratic Programming Optimization with Feature Selection for Non-linear Models // Lobachevskii Journal of Mathematics, 2018, 39(9) : 1179-1187. \href{https://rdcu.be/bfR32}{PDF}. 
\item  Motrenko A.P., Strijov V.V. Multi-way feature selection for ECoG-based brain-computer interface // Expert Systems with Applications, 2018, 114(30) : 402-413. \href{http://strijov.com/papers/MotrenkoStrijov2017ECoG_HL_2.pdf}{PDF}. 
\item  Eliseyev A., Aksenova T. Stable and artifact-resistant decoding of 3D hand trajectories from ECoG signals using the generalized additive model. Journal of neural engineering. – 2014.
\end{itemize}

\end{document}
