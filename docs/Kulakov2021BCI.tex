\documentclass{article}
\usepackage[utf8]{inputenc}
\usepackage{setspace}
\usepackage{hyperref}
\usepackage[T2A]{fontenc}	
\usepackage{amssymb}
\usepackage[english, russian]{babel}
\title{Выбор согласованных моделей для построения нейроинтерфейса}
\author{Кулаков Ярослав}
\date{February 2021}
\doublespacing

\begin{document}

\maketitle

\section{Abstract}
В работе исследуется задача построения нейро-компьютерного интерфейса (BCI). Рассматривается задача моделирования сигнала, снимаемого с конечностей при активности мозга. Предлагается применить методы снижения размерности исходного пространства, так как значения сигнала сильно скоррелированы.

Проводится анализ прогноза и латентного пространства, получаемых парой гетерогенных моделей.

Эксперементальные результаты подтверждают, что предлагаемый метод улучшает качество предсказаний модели.

\section{Introduction}
В настоящее время все больше задач перекладывается на плечи машин, машинного обучения. В то же время человеческий мозг остается неизученным и трудно прогнозируемым объектом. 
В работе предлагается исследовать разные модели машинного обучения: линейные и нелинейные нейросети. Оценивается качество, устойчивость и сложность рассматриваемых моделей. Для получения нескоррелированных, но информативных признаков, решается задача снижения размерности исходного пространства. В статье (Катруца, Стрижов, 2017) даны обширные сравнения алгоритмов снижения размерности: QPFS с LARS, Lasso, Stepwise, Ridge и отбор признаков с генетическим алгоритмом. Quadratic Programming Feature Selection (QPFS) превосходит конкурентов и этот метод можно адаптировать для нашей задачи. Так же проводится сравнение с методами PLS, PCA, других нелинйных моделей.  При решении задачи выбора признаков, одновременно оптимизируются две задачи: минимизируется корреляция между признаками и максимизируется информативность признаков по отношению к таргету. Задача осложняется тем, что признаки и таргеты имеют разную природу. \par
В результате получен устойчивый пайплан модулирования сигнала в конечности от карты активности мозга, состоящий из этапов:
\begin{itemize}
    \item Построение латентного пространства меньшей размерности, с минимальной корреляцией признаков и максимальной информативностью.
    \item Построение прогностической модели в полученном пространстве.
     \item Восстановление обратной зависимости для предсказания активности мозга.
\end{itemize}

\section{Problem statement}
Рассматривается датасет $(X, Y).$ Данные содержат записи о траектории движения руки в 3х-мерном пространстве и ECoG сигнала. Датасет состоит из 20-ти записей двух обезьян, которые пытались достать кусочек еды правой рукой. ECoG сигнал снимался с 64х электродов, частотой 1кГц. Чтобы сформировать тензор признаков, каждая эпоха ECoG была сопоставлена с временно-частотно-пространственным пространством с помощью непрерывного волнового преобразования (CWT). \par
Снижение размерности. Задача состоит в поиске функций $\phi: X^{n\times m} \rightarrow X^{n\times k}$ для объектов и функции  $\psi: Y^{n\times p} \rightarrow X^{n\times q}$ для кодирования целевых переменных. Причем $k < m, q < p.$ Полученные матрицы являются матрицами представлений в латентном пространстве.
\par
Рассматриваются линейные и нелинейные модели. Линейные модели менее подвержены проклятию размерности и слабее переобучаются, а нелинейные методы способны уловить сложные закономерности. Чтобы объединить преимущества обоих методов рассматриваются generalized linear models, additive models и их комбинация --- GAM. \par
Линейная модель задается в виде $\hat y = X^T\theta + \theta_0,$ где $X$ --- матрица объект-признаков, а $\theta$ --- вектор параметров модели. Предполагается, что истинная зависимость так же является линейной, с шумом, распределенным нормально. \par
GLM --- обобщение линейной регрессии, в котором мы можем применять разные функции к $y$, а так же предполагать разные его распределния. \par
AM --- еще один способ по внедрению нелинейности в модель. $\hat y = \theta_0 + \sum \theta_i f_i(x_i)$. \par
Используемые метрики и критерии качества: $MSE(||y-\hat y||_2^2 ), MAE(||y-\hat y||_1), MAPE(\frac{1}{n}\sum \frac{|Y_i-\hat Y_i|}{Y_i})$.  \par
Соответственно задача ставится как минимизация этих Loss-function: $L(X, Y, \Theta) \rightarrow min.$


\section{Literature}
\renewcommand\labelitemi{$\textasteriskcentered$}
\begin{itemize}
\item  Яушев Ф.Ю., Исаченко Р.В., Стрижов В.В. Модели согласования скрытого пространства в задаче прогнозирования  Системы и средства информатики, 2021, 31(1). \href{http://strijov.com/papers/Isachenko2020CanonicCorrelation.pdf}{PDF}.
\item  Исаченко Р.В. Выбор модели декодирования сигналов в пространствах высокой размерности. Рукопись, 2021.\href{https://github.com/r-isachenko/PhDThesis/raw/master/doc/Isachenko2021PhDThesis.pdf}{PDF} 
\item  Исаченко Р.В. Выбор модели декодирования сигналов в пространствах высокой размерности.  \href{http://strijov.com/papers/IsachenkoVladimirova2018PLS.pdf}{PDF}. 
\item  Isachenko R.V., Strijov V.V. Quadratic Programming Optimization with Feature Selection for Non-linear Models // Lobachevskii Journal of Mathematics, 2018, 39(9) : 1179-1187. \href{https://rdcu.be/bfR32}{PDF}. 
\item  Motrenko A.P., Strijov V.V. Multi-way feature selection for ECoG-based brain-computer interface // Expert Systems with Applications, 2018, 114(30) : 402-413. \href{http://strijov.com/papers/MotrenkoStrijov2017ECoG_HL_2.pdf}{PDF}. 
\item  Eliseyev A., Aksenova T. Stable and artifact-resistant decoding of 3D hand trajectories from ECoG signals using the generalized additive model. Journal of neural engineering. – 2014.
\end{itemize}


\end{document}
